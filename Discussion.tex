%%% Time-stamp: <2023-08-29 17:23:40 vladimir>
%%% Copyright (C) 2019-2023 Vladimir G. Ivanović
%%% Author: Vladimir G. Ivanović <vladimir@acm.org>
%%% ORCID: https://orcid.org/0000-0002-7802-7970

\chapter{Discussion}\label{ch:discussion}

\begin{comment}
  With sheer repetion, and in the absense of evidence, a myth about K-12 education has taken hold: American public schools are abject failures. Something must be done to reign in the rapacious unions who protect and coddle incompetent teachers. Something must be done about lazy administrators who block progress. Something must be done to give back to parents control over their children's education. And that something is charter schools.

  Rocketship is one of the most successful charter school chains in the United States, but their success is not in educating elementary school children. Case in point: In August 2023, the Fort Worth Star-Telegram reported that only 23\% of Rocketship's students met state standards in reading and language arts \parencite{Allen.Ruiz2023} compared to 53\% statewide \parencite{TexasEducationAgency2023}.

  Instead, Rocketship's success is in making money.
\end{comment}
This dissertation's research question is ``Is Rocketship primarily a moneymaking operation that uses public education funds to finance the acquisition of real estate?'' In order to answer that question, my findings need to establish a convincing argument that
\begin{enumerate}
  \item Rocketship is profitable, and
  \item profitability is the most plausible explanation for how they've structured themselves and how they operate.
\end{enumerate}

The first criterion can be established by scrutinizing Rocketship's financial statements. Establishing the second is less straightforward. Ideally, a document authored by Rocketship's founders where that is stated would be sufficient. Unfortunately, no such document exists, and surprisingly, it is not explicitly stated by any charter school or pro-charter school advocacy organization.

The nearest there is to a charter school agenda and rationale is the 300+ page report from GSV (Global Silicon Valley) Advisors \citetitle{Moe.etal2012} by \citeauthor{Moe.etal2012}. GSV Advisors are investment advisors to the \emph{digerati} of Silicon Valley, and their focus is making money. As an indication of just how much \citetitle{Moe.etal2012} focus on money, it is sufficient to note that the titles of eight out of the nineteen sections of that report are explicitly about markets and investments. Rocketship is one of the dozen or so ``education innovators'' given a thumbnail sketch. %\parencite[228]{Moe.etal2012}

\section{Judging Case Studies}\label{sec:case-studies}

\begin{comment}
\subsection{Validity}
  \subsection{Reliability}
  \subsection{Limitations}
  \section{Evaluating the Results}
  \section{Rival Explanations}
  \section{Future Research}
\end{comment}



%%% Local Variables:
%%% mode: latex
%%% TeX-master: "Rocketship_Education-An_Exploratory_Public_Policy_Case_Study"
%%% End:
