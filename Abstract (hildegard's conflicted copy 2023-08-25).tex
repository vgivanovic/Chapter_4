%%% Time-stamp: <2022-07-31 21:18:47 vladimir>
%%% Copyright (C) 2019-2022 Vladimir G. Ivanović
%%% Author: Vladimir G. Ivanović <vladimir@acm.org>
%%% ORCID: https://orcid.org/0000-0002-7802-7970

\addcontentsline{toc}{chapter}{Abstract}
\begin{center}
\thetitle%
\end{center}
\begin{abstract}
\noindent{}This dissertation is an exploratory case study of the finances of the Rocketship charter school chain, especially those related to real estate. Rocketship is a  not-for-profit charter management organization, one of the first in Santa Clara County, California. This study seeks to determine if the financial transactions related to Rocketship charter schools yield profits for investors, despite Rocketship itself being a non-profit entity, and if they do, how and where do they do so. In order to characterize fairly and completely the profits of Rocketship Education itself and Rocketship-related entities, this study uses publicly available documents to track money flowing in and out of Rocketship and related entities, for example, the various Launchpad Development companies. Using data from initial and renewal charter petitions, annual budget documents, filings with county, state and federal government agencies, bond prospectuses, tax credit programs, state and federal grants, plus data from publicly available datasets, this study derives an estimate of Rocketship's profitability. It found that [Results TBD].  [Conclusion TBD]. These results, it is hoped, will serve to inform local, state, and federal legislatures when they establish public policy for charter schools, not only in California, but throughout the United States.\bigskip

\noindent\textit{Keywords}: Rocketship Education, charter management organization, privatization, charter finances, education public policy, profit, real estate, bonds, venture funds, philanthrocapitalism
\end{abstract}

%%% Local Variables:
%%% mode: latex
%%% TeX-master: "Rocketship_Education:_An_Exploratory_Public_Policy_Case_Study"
%%% End:
