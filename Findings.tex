% Time-stamp: <2023-10-26 22:47:06 vladimir>
% Copyright (C) 2019-2023 Vladimir G. Ivanović
% Author: Vladimir G. Ivanović <vladimir@acm.org>
% ORCID: https://orcid.org/0000-0002-7802-7970

\begin{comment}
  Explaining the real estate-related finances of Rocketship Education is the heart of this dissertation. Where do Rocketship's revenues come from? Where are they spending that revenue? Are there investors who make money off of Rocketship? And, critically, if Rocketship takes in more money than it spends on education, where does that money go?
\end{comment}
\chapter{Findings}
\label{ch:findings}

This chapter presents data found using the approach outlined in \prettyref{ch:methods} with the goal of answering this dissertation's research question: ``Has Rocketship structured itself and its finances, to earn a return to investors, focusing especially on real estate transactions, and if so, how?''

The first section presents Rocketship's corporate structure, a structure that separates Rocketship schools from Rocketship facilities. The next section, \prettyref{sec:location-and-property-info}, details what facilities Rocketship has, where those facilities are located, when they were acquired, and what real estate rights Rocketship has in those properties. Then, given Rocketship real estate, the third section characterizes the finances of Rocketship that are used to fund those properties. The penultimate section reviews what gaps, anomalies and discrepancies were found in Rocketship's financial data. The final section, \prettyref{sec:issues_equality_equity}, looks briefly at issues of fairness.

Note that Rocketship financial data is not available for all years, and starting in 2014, the data is for all of Rocketship Education, including schools in Wisconsin, Tennessee, and Washington, D.C.


\section{Rocketship's Corporate Structure}
\label{sec:RSED-corporate-structure}\indent%

One of the original four members of Rocketship's board of directors was Eric Resnick, a specialist in real estate finance. He was expected to ``provid[e] a deep understanding of financial management and real estate transactions'' \parencite[13]{Danner2006}, so it appears that Rocketship's corporate structure was designed with real estate transactions in mind, and from the start, Rocketship has kept schools and their facilities separate. This structure is diagrammed in \prettyref{fig:corporate-structure} on p.\pageref{fig:corporate-structure} for Rocketship facilities in Santa Clara County.

\begin{figure}[ht]
  \centering\scriptsize
  \caption[Rocketship's Corporate Structure for Santa Clara County Facilities]{\emph{Rocketship's Corporate Structure for Santa Clara County Facilities}}\label{fig:corporate-structure}
  \sffamily
  \begin{forest}
    for tree={grow'=east, folder, draw, align=left}
    [ \textbf{Rocketship Education (RSEA)}, baseline
    [ Launchpad Development Company (LDC), xshift=2em
    [ \textit{Launchpad (LP)}, xshift=4em ]
    [ \textit{Launchpad Development One LLC (LLC1) RMS facilities}, xshift=4em ]
    [ \textit{Launchpad Development Two LLC (LLC2) RSSP facilities}, xshift=4em ]
    [ \textit{Launchpad Development Threee LLC (LLC3) RLS facilities}, xshift=4em ]
    [ \textit{Launchpad Development Four LLC (LLC4) ROMO facilities}, xshift=4em ]
    [ \textit{Launchpad Development Five LLC (LLC5) RDP facilities}, xshift=4em ]
    [ \textit{Launchpad Development Eight LLC (LLC8) RSA facilities}, xshift=4em ]
    [ \textit{Launchpad Development Ten LLC (LLC10) RSK facilities development}, xshift=4em ]
    [ \textit{Launchpad Development Eleven LLC (LLC11) RBM facilities}, xshift=4em ]
    [ \textit{Launchpad Development Twelve LLC (LLC12) RFZ facilities}, xshift=4em ]
    [ \textit{Launchpad Development Sixteen LLC (LLC16) RRS facilities}, xshift=4em ]
    ]
    [ Rocketship Support Network (RSN), xshift=2em ]
    [ Rocketship Mateo Sheedy Elementary (RSM), xshift=2em ]
    [ Rocketship Sí-Se-Puede Academy (RSSP), xshift=2em ]
    [ Rocketship Los Sue (RLS), xshift=2em ]
    [ Rocketship Mosaic Elementary (ROMO), xshift=2em ]
    [ Rocketship Discovery Prep (RDP), xshift=2em ]
    [ Rocketship Alma Academy (RSA), xshift=2em ]
    [ Rocketship Brilliant Minds (RBM), xshift=2em ]
    [ Rocketship Spark Academy (RSK), xshift=2em ]
    [ Rocketship Fuerza (RFZ), xshift=2em ]
    [ Rocketship Rising Stars (RRS), xshift=2em ]
    ]
  \end{forest}
\end{figure}

The parent corporation, Rocketship Education, Inc. (RSED), a 501(3)(c) public benefit corporation, was formed in California on February 16, 2006. RSED owns all the Rocketship schools and Launchpad Development Company, a 509(a)(3) nonprofit public benefit corporation.\footnote{A 509(a)(3) corporation is a ``charity that carries out its exempt purposes by supporting other exempt organizations, usually other public charities'' and ``has a relationship with its supported organization sufficient to ensure that the supported organization is effectively supervising or paying particular attention to the operations of the supporting organization.'' \parencite[accessed 29 Sep 2023]{IRS2023}}. RSED plus the schools plus Launchpad Development Company is known as Rocketship Education and Its Affiliates (RSEA).

Launchpad Development Company owns one LLC for each school's facility, generally named ``Launchpad Development <number> LLC''. In addition, Rocketship has two functional divisions:
\begin{itemize}
  \item Rocketship Support Network (RSN) which provides resources for management, back-office support, and organizational strategy to Rocketship schools
  \item Launchpad (LP) which provides investment and asset management, and administrative services to Launchpad LLCs
\end{itemize}

\begin{comment}
  This separation between the operation of schools from the funding of their facilities raises the question of why Rocketship has chosen this structure. Some possibilities are:
  \begin{itemize}
    \item The owners of a school's facilities are not schools and thus are not directly subject to California's Education Code.
    \item Facility owners are probably insulated from legal action against schools.
    \item RSED can charge schools management and facility fees that might not be allowed for schools themselves.
    \item When a charter school closes, the real estate assets (land + buildings + improvements) are not owned by the now closed charter school, but by a separate entity (Launchpad Development One/Two/Three/\ldots) which is owned by another entity (Launchpad Development Company) that itself is owned by yet another entity (Rocketship Education, Inc.). The result is that no assets need to change hands if a school closes.
  \end{itemize}
\end{comment}

Rocketship gave four reasons for this corporate organization:
\begin{enumerate}
  \item The need to eliminate RSED liability. Without Launchpad and its LLC's, RSED is taking on several liabilities
  \begin{itemize}
    \item developing financing deals
    \item lawsuit related to CEQA
    \item financial risk from financing
  \end{itemize}
  With Launchpad and LLCs, RSED will have no liability associated with real estate.
  \item The need to manage RSED's cash flow fluctuates whenever a new school is financed. These fluctuation are large and lead to unnecessary speculation.
  \item The need to allow RSED to focus on ``Great Schools'' and to let Launchpad focus on building ``Great Sites''.
  \item The need to increase the market for developers of charter facilities.
\end{enumerate}
At a Board offsite on 23 Jun 2009, Rocketship expanded on these reasons in a presentation, \textcite{RSED2009}. 

\section{Rocketship Locations and Property Information}
\label{sec:location-and-property-info}\indent%

Before the formation of Launchpad, the Rocketship board chose sites for its schools according the following criteria:
\begin{itemize}
  \item Location: Within 1 mile of a PI or otherwise low performing School, Qualifies for New Market Tax Credit Criteria (75\% FRL)
\end{itemize}
In September 2009, they added
\begin{itemize}
  \item Financials: Less than \$8M for 30 years at 5\% interest.
  \item Enrollment: For a school with 450 K-5 students, at least 3× that number within 1 mile or compensating interest from families outside the 1 mile radius.
\end{itemize}
These selection criteria appeared very early on (September 2009)\parencite{RSED2009b} and they demonstrate that Rocketship was aware of the NMTC criteria and choose schools which would qualify.

Details of the Rocketship schools listed in \prettyref{tab:locations} are given in \prettyref{appx:rocketship-property-info} on p.\pageref{appx:rocketship-property-info}.

\begin{table}[hbt]
  \caption[Rocketship Property Information]{\textit{Rocketship Property Information}}\label{tab:locations}\SingleSpacing%
  \begin{tabular}{lll}
    \toprule
    School          & Address                               & Property Information \\
    \midrule
    Mateo Sheedy    & 788 Locust St., San José, CA 95110    & \prettyref{sec:mateo-sheedy-info} \\
    Sí Se Puede     & 2249 Dobern Ave, San José, CA 95116   & \prettyref{sec:sí-se-puede-info} \\
    Los Sueños      & 331 S. 34th St, San José, CA 95116    & \prettyref{sec:los-suenos-info} \\
    Discovery Prep  & 370 Wooster Ave, San José, CA 95116   & \prettyref{sec:discover-prep-info} \\
    Mosaic          & 950 Owsley Ave, San José, CA 95122    & \prettyref{sec:mosaic-info} \\
    Brilliant Minds & 2960 Story Rd, San José, CA 95127     & \prettyref{sec:brilliant-minds-info} \\
    Alma Academy    & 198 West Alma Ave, San José, CA 95110 & \prettyref{sec:alma-academy-info} \\
    Spark Academy   & 683 Sylvandale Ave San José, CA 95111 & \prettyref{sec:spark-academy-info} \\
    Fuerza          & 70 S. Jackson Ave, San José, CA 95116 & \prettyref{sec:fuerza-info} \\
    Rising Stars    & 3173 Senter Road, San José, CA 95111  & \prettyref{sec:rising-stars-info} \\
    \bottomrule
  \end{tabular}
\end{table}


\section{Rocketship's Finances}
\label{sec:rocketship_finance}\indent%

Financing charter schools in California is more complicated than the financing of traditional public schools because charters need to obtain often independent facilities from the public school district in which they are located.
\prettyref{tab:charter-school-financing-options} on p.\pageref{tab:charter-school-financing-options} describes what facilities financing options a charter school has compared to a traditional public school. Note that ending up with facilities that satisfy a school's needs may require the purchase of land, the construction of new facilities, or the modification of existing facilities in addition to operating those facilities. Each of these alternatives may have different potential financing options.

To illustrate the variety of financing options that may be used, Rocketship states that they used three different financing options for nine schools as of 2015.
\begin{quotation}\noindent
Launchpad successfully financed four of the nine Bay Area Rocketship projects with New Market Tax Credits, four projects by issuing long term tax exempt bonds, and one project through short term private financing. \sourceatright{\parencite[161]{Alexander2015a}}
\end{quotation}

Rocketship also prepared a detailed financial model, \textit{Current RSED Financial Model 061909} \parencite{RSED2009a}, that extrapolates some data to 2045. Listing the sheet names gives an indication of the depth of the analysis that Rocketship put into creating this financial model. 
\begin{table}[ht]
    \begin{tabular}{lll}
      Summary & RSED National & National Systems Costs \\
      All Schools Roll-up & All Facilities Roll-up & Balance Sheets-v1\\
      RS1 & Locust LLC & RS2 \\
      Dobern LLC & RS3 & RS3 LLC \\
      RSGen & RSGen LLC & Balance Sheets \\
      RSED Growth Plan & Buildings Book Balance & Buildings Depreciation \\
      Facilities Loan Balance & 0910 RMS & 0910 RS2 \\
      0910 Nat & Launch Pad & ITAchievementOps \\
      Change Log & Network Statistics Summary & All Schools Analysis\\
      0910 Rev Compare & RSED \#1  - NSLP & RSED \#2 - NSLP\\
      0910 SPED &  Abacus Notes \\
  \end{tabular}
\end{table}

Although the mass of documents asked for by FCMAT is much larger, the depth of analysis by Rocketship is significantly greater. The entire spreadsheet is available at \url{https://docs.google.com/spreadsheets/d/1e5j8nn2Ofg6l5BlOaPi_qcByGH_OAt232RrvTkoJy2Q}. Rocketship maintains an archive of its board agendas \& materials, and minutes at \url{https://www.rocketshipschools.org/about/board-of-directors/board-agendas-archive/}, but it has removed agendas \& material and minutes prior to 2017. Any meeting material used here prior to that date was collected before Rocketship removed those files, i.e. they are no longer available on Rocketship's web site.

Also revealing of Rocketship's early financial thinking is \citetitle{FinNarr2010} \parencite{FinNarr2010}. In it, Rocketship describes the parameters of a typical school (see the sheets ``RSGen'' in \textcite{RSED2009a}) from one year before opening to year 10. Some noteworthy observations:
\begin{itemize}
  \item The very first sentence is, ``Due to Rocketship's "Hybrid" educational model, each Rocketship school reaches breakeven in year 1 of operation.''\parencite[1]{FinNarr2010} epitomizes Rocketship's focus on financials to the detriment of scholatic achievement.
  \item Rocketship predicts that student demographics will be 70\% Free and Reduced Lunch in years 1–8+, 50\% below the federal poverty level, likewise in years 1–8+, and English Language Learners (ELL) dropping from 70\% in year 1 to 50\% in year 8+.
  \item Rocketship expects each school to receive \$500K total in grants from the Walton Family Foundation and Reed Hastings, plus \$600K in federal startup grants through Title V.\parencite[2]{FinNarr2010}.
  \item The first seven schools are to pay 25\% of revenue (less food service sales and reimbursements) in management fees in the year before opening (year -1), dropping to 15\% in years 0–3+. Facilities fees start in year 1 and are 20\% of revenue (less food service sales and reimbursements). For schools 8+, there are no fees in years -1 through 1.\parencite[6]{FinNarr2010}
\end{itemize}

\subsection{Charter School Financing Options}
\label{sec:charter-school-financing-options}\indent%

\begin{table}[ht]
  \caption[Charter School Financing Options]{\textit{Charter School Financing Options}}
  \label{tab:charter-school-financing-options}%
  \SingleSpacing%
  \begin{tabular}{lccl}
    \toprule
    \textbf{Type}        & \multicolumn{2}{c}{\textbf{Available to}}  & \textbf{Notes}\\
                         & \textbf{TSPs} & \textbf{Charters}          & \\
    \midrule
    \multicolumn{4}{l}{\textit{State funding}}  \\
    \midrule
    LCFF                 & Yes  & Yes                        & State minimum guarantee: ADA + adjustments\\ 
    Local property tax   & Yes  & Yes                        & Reduces LCFF amount\\
    Categorical programs & Yes  & Yes                        & \multirow[t]{2}{3in}{All state funding outside of LCFF is\\
    categorical. Some federal programs exist.} \\
    \\
    \multicolumn{4}{l}{\textit{Local funding}}\\
    \midrule
    Local parcel tax     & Yes  & No                         & Established by district-wide election\\
    Bonds                & Yes  & Yes                        & \multirow[t]{2}{3in}{Public schools: district election \\
    Charters: private or gov't sponsored}\\
    \\
    \multicolumn{4}{l}{\textit{Federal, state, or private funding}}\\
    \midrule
    Private grants       & Yes & Yes                         & Much more common with charters\\
    Venture fund loans   & No  & Yes                         & Often using New Market Tax Credit program\\
    Rent subsidies       & No  & Yes                         & By the state (SB740)\\
    COVID-19 PPP loans   & No  & Yes                         & Paycheck Protection Program loan becomes a grant\\
    \bottomrule
  \end{tabular}
\end{table}

The first three sources of financing listed in \prettyref{tab:charter-school-financing-options} are considered ordinary revenue which are available to both public and charter schools, although the amounts and timing of the distributions vary. The remainder are not necessarily present for a given charter school or public school district.

\subsubsection{LCFF}
\label{sec:lcff}\indent%

The Local Control Funding Formula is the principal way California funds both charter schools and public schools. County Offices of Education receive from the California Department of Education funds calculated using the Local Control Funding Formula (with adjustment) and those funds are distributed to all public school districts in the county. In turn, public school districts pass through an amount also calculated using the LCFF calculation.

All schools have the same base grant which varies by grade span. If a school, charter or public, has students who are in one or more of the following categories (1) eligible for free or reduced price meals (FRPM), (2) are English Learners (EL), or (3) are foster youth, the school receives a supplemental grant of 20\% of its base grant for each such student. If the qualifying population of students\footnote{These are unartfully called ``unduplicated pupils'' because schools do not get extra money for students in more than one category, as they should.} exceeds 55\% of the total number of students, a school receives a concentration grant of 65\% of the base grant for every student above the 55\% threshold.

All Rocketship schools are located in high-poverty areas and all have more than 55\% of their students in at least one of the three categories that qualify for concentration grants in addition to supplemental grants. 

\subsubsection{Property Taxes}
\label{sec:property-taxes}\indent%

In California, commercial and privately owned properties are taxed. School districts receive 40\% of the property tax collected from properties in their district and this tax replaces an equal portion of LCFF revenue. (If a district's property tax revenue exceeds what they would have gotten in LCFF funding, they receive no LCFF funding. These districts are called \textit{community-funded districts}, previously known as \textit{basic aid districts}.) Note that the amount that districts pass through is independent of how much property tax is collected; it is always the full LCFF amount.

\subsubsection{Parcel Taxes}
\label{sec:parcel-taxes}\indent%

Traditional public school district may assess a non-\textit{ad valorem} tax, usually a per parcel tax\footnote{A 2023 court decision allowed a tax based on square footage because it is also a non-\textit{ad valorem} tax.} if voters approve. Charter schools do not have taxing authority, so they may not assess parcel taxes. Public school districts may agree to share some portion of their parcel tax revenue with charter schools within their boundaries, but are not required to do so. Rocketship has no parcel tax revenue.

\subsubsection{Bonds}
\label{sec:bonds}\indent%

Bonds, as far as educational institutions are concerned, come in just a few forms\footnote{A complete discussion of the various forms, constraints on, and capabilities of governmental debt in California can be found in \citetitle{CDIAC2023} \parencite{CDIAC2023}}. These are:

\begin{itemize}
  \item General Obligation (GO) bonds are backed by the full faith and credit of the issuer, here a public school district or a charter school. Normally, bonds are secured by assets owned by the borrower, such as real estate, personal property (e.g. an airplane or an oil well), or some other physical asset. Lenders (the purchasers of a bond) are naturally reluctant to lend based on an ephemeral asset like a revenue stream because of the chance that the revenue stream might dry up. The solution for charter schools is conduit borrowing described below.

  Unlike public school districts that can pass a bond measure based on the value of the entire district's assessed property, charter schools have either no real property (if they are leasing) or a very small amount (if they own their facilities), so even if they were allowed to put a bond measure to the voters, the GO debt limit of 1¼\% of their facility's assessed value would provide very limited funds. For example, an \$80M valuation would be required to be able to issue a \$1M bond. 

  \item Tax and Revenue Anticipation Notes (TRANs) and Revenue Anticipation Notes (RANs) are backed by specific forms of revenue. 
  \item Conduit Revenue Bonds are issued by and are an obligation of a government agency (the conduit) that is neither the borrower nor the purchaser. The government entity or agency functions as a conduit between a borrower and the purchaser of the bond (i.e. the lender). Typically, the conduit is a state agency that administers an offering by loaning to the borrower money it has received from another government agency, typically the Federal government. The borrower's payments to the conduit are sized to meet the payments needed to repay the purchaser of the debt.
\end{itemize}

This dissertation's Data Dashboard\footnote{\url{https://docs.google.com/spreadsheets/d/1bnBIUkx7EPZU2UEUxi5M4BwkSgVjmKYVaZTnBZgIq8I}} shows that Rocketship took out a number of loans. In the years ending 2008 through 2011, Rocketship borrowed at least fifteen times before actually floating a bond, Series 2011A \& B in 2012.

\subsubsection{Private Grants}
\label{sec:private-grants}\indent%

Rocketship lists a total of \$78,387,835 as ``Contributions'' from 2010 through 2022 (see line 11 of \prettyref{tab:consolidated_activities} in Appended E on \pageref{tab:consolidated_activities}). Unfortunately, the details of what those contributions actually comprise are not available. 

\begin{comment}
  This sum likely understates the true value of private grants primariy due to the conversion of debt or loans into grants as seen in \prettyref{appx:debt_2010-22} on\~p.\pageref{appx:debt_2010-22}.
\end{comment}

\subsubsection{Venture Fund Loans}
\label{sec:venture-fund-loans}\indent%

\subsubsection{Rent Subsidies}
\label{sec:rent-subsidies}\indent%

\subsubsection{COVID-19 PPP Loans}
\label{sec:covid-19-ppp-loans}\indent%

\subsection{Rocketship Financial Documents}
\label{sec:rocketship-financial-docs}\indent%

Every year, as required by law, Rocketship issues an independently audited financial statement for the preceding school year. Rocketship, rather than issuing a separate financial statement for each of its affiliates, consolidates them into a single document, typically called \textit{Rocketship Education, Inc. and Its Affiliates Consolidated Financial Statements and Supplementary Information Year Ended June 30, \#\#\#\#}. Four annual financial statements are reported:

\begin{itemize}
  \item Financial Position, which corresponds to a business's balance sheet
  \item Activities, which corresponds to a business's income statement
  \item Cash Flows, which corresponds to a business's cash flow statement
  \item Functional Expenses, which is usually only used by non-profits
\end{itemize}

The four different financial statements for the years 2010–2022\footnote{The years ending 2008 (2007–2008) and 2009 (2008-2009) have not been included in some summaries because they were different from all the other years. The year 2009 included 2008, and 2008 was restated in 2022.} have been collected and the data summarized. These summaries appear in \prettyref{tab:consolidated_financial_position}, \prettyref{tab:consolidated_activities}, \prettyref{tab:consolidated_cash_flows}, and \prettyref{tab:consolidated_functional_expenses}, in Appendices D – G, and online in this dissertation's \textit{Data Dashboard}, a Google spreadsheet.
\footnote{\url{https://docs.google.com/spreadsheets/d/1c4akEKFj9bmVfLFQwi7ewMifSjRbrw5xpjh_UjO4oYY/edit?usp=sharing}}

Rocketship's net assets from 2010 to 2022 have always been positive as seen in \prettyref{tab:net_assets_annual_change}, although they have risen in some years and fallen in others, sometimes considerably so.
\begin{table}[ht]
  \caption[Net Assets, 2010–2022]{\textit{Net Assets, 2010–2022}}
  \label{tab:net_assets_annual_change}
  \begin{tabular}{rrr}
    \toprule
    \textbf{Year} & \textbf{Net Assets} & \multirow[t]{2}{0.6in}{\textbf{Annual \\
    Increase}}\\
    \\
    \midrule
    2010 &   \$2,218,964	&            \\
    2011 &   \$9,212,140	&   315.16\% \\
    2012 &  \$11,933,099	&    29.54\% \\
    2013 &  \$15,881,210	&    33.09\% \\ 
    2014 &  \$13,356,528	&   -15.90\% \\
    2015 &  \$10,562,747	&   -20.92\% \\
    2016 &  \$16,931,464	&    60.29\% \\
    2017 &  \$17,536,163	&     3.57\% \\
    2018 &  \$20,883,606	&    19.09\% \\
    2019 &  \$24,084,572        &    10.12\% \\
    2020 &  \$24,617,294        &     2.21\% \\
    2021 &  \$38,231,318	&    55.30\% \\ 
    2022 &  \$33,442,645        &   -12.53\% \\
    \bottomrule
  \end{tabular}
\end{table}

\subsection{Debt}
\label{sec:debt}\indent%

Rocketship has borrowed over 50 times since its founding in 2006\footnote{Full details of Rocketship's borrowings are in this dissertation's Google spreadsheet (see the previous footnote) in the tab \textit{Dashboard} starting on lines 10–63.} 
and their annual, consolidated financial statements provide debt summaries starting in 2012. The totals are shown in \prettyref{tab:total_debt} below.

The annual increase (or decrease) in debt year-over-year is shown in \prettyref{tab:total_debt} One can immediately observe that the changes in debt year-over-year are quite pronounced. They range from a low of 86\% to a high of 155\% with an average of just over 100\% per year. (An increase of 86\% means that the absolute amount of debt decreased. An increase of 155\% means that debt increased one and a half times the previous year.)

\begin{table}[ht]
  \caption[Total Debt, 2012-2022]{\textit{Total Debt, 2012-2022}}
  \label{tab:total_debt}
  \begin{tabular}{rrr}
    \toprule
    \textbf{Year} & \textbf{Total Debt} & \multirow[t]{2}{0.6in}{\textbf{Annual \\
    Increase}}\\
    \\
    \midrule
    2012 & \$47,046,048 & \\
    2013 & \$57,078,166 & 121.32\% \\
    2014 & \$88,383,082 & 154.85\% \\
    2015 & \$75,904,098 &  85.88\% \\
    2016 & \$104,857,696 & 138.14\% \\
    2017 & \$136,652,562 & 130.32\% \\
    2018 & \$129,391,897 &  94.69\% \\
    2019 & \$163,598,844 & 126.44\% \\
    2020 & \$168,701,124 & 103.12\% \\ 
    2021 & \$196,416,045 & 116.43\% \\
    2022 & \$186,550,566 &  85.89\% \\
    \bottomrule
  \end{tabular}
\end{table}

\subsubsection{Financing Debt}
\label{sec:financing_debt}\indent%

\paragraph{Letters of Credit}
\paragraph{Municipal Bonds}
\paragraph{Conduit Bonds}

\subsection{Non-Debt Financing}
\label{sec:non_debt_financing}\indent%

\subsubsection{Donations and Grants}
\label{sec:donations_grants}\indent%

\subsubsection{Forgiveness of Debt}
\label{sec:forgiveness_debt}\indent%

\subsubsection{Loans}
\label{sec:loans}\indent%

\subsubsection{Venture Funds}
\label{sec:venture_funds}\indent%

\subsection{The New Markets Tax Credit (NMTC) Program}
\label{sec:NMTC}\indent%

The New Markets Tax Credit (NMTC) program is one of six programs offered by the Community Development Financial Institutions Fund (CDFI Fund) of the U.S. Department of the Treasury. In essence, the NMTC program offers a 39\% tax credit on qualified investments in ``Low-Income Communities''. The credits are spread out over 7 years: 5\% of the amount invested for the first 3 years, and 6\% for the remaining 4 years. These credits can be applied to federal income taxes due from other investments. 

Charter schools operating in economically depressed areas qualify for tax credits. 

It is incentives like these incentives make the NMTC popular, coupled with reduced risk. The tax credit investment is nearly without risk because the tax credit is guaranteed as long as the charter school remains open. If the school stays open for seven years, the risk is nearly zero.

An example will help make it clear how the NMTC works. Suppose we have a high wealth individual with a marginal tax rate of 37\% (the highest bracket), and suppose that this high wealth individual gets a return 10\% per annum on their investments.

So, suppose this individual has \$2,351,000 to invest. He could divide that amount into a \$1,000,000 NMTC investment and \$1,351,000 investment in something other than a qualified NMTC investment. \\

In the NMTC case, every year for the first three years, the investor has \$1,000,000 $\times$ a 10\% return $\times$ 37\% income tax = \$37,000 tax due on \$100,000 profit, The investor gets back \$1,000,000 + (\$1,000,000 × 10\%) - (\$1,000,000 × 0.37) = \$1,063,000 after taxes, in addition to a \$1,000,000 × 5\% = \$50,000 tax credit. The last four years, the tax credit rises to \$60,000. \\
  
In the non-qualifying investment, the investor has \$1,351,000 $\times$ 10\% return $\times$ 37\% income tax = \$50,000 tax due, which is exactly equal to the tax credit of the NMTC case. The investor also has the return on the investment whose tax due was equal to the NMTC tax credit: \$,1351,000 + (\$1,352,000 $\times$ 10\$) = \$1,486,000.\\

Combining the returns from the NMTC case with those from the non-NMTC case, we get back \$1,063,000 + \$1,486,000 = \$2,549,000 for an investment of \$2,351,000, a 8.4\% net return after taxes, with reduced overall risk.
  
  In general, the after tax return of an investment with a pre-tax return of 10\% for a high wealth individual is 6.3\%, so the NMTC after tax return is 33.3\% higher with much reduced risk.


\section{Gaps,  Anomalies, and Discrepancies}
\label{sec:gaps_anomolies_discrepencies}\indent%

This section is concerned with what wasn't found during our investigation. Gaps would be where data was expected, but none was found. Anomalies are where data was found, but it differed from what was expected, and discrepancies are where data was found but conflicted with other data which was found.

In an enterprise as large as Rocketship is now (with a \$190M+ budget in 2022), there are bound to be unintentional gaps, anomalies, and discrepancies without any implication of nefarious intent. Further, as Berman and Knight emphasize, accounting involves making assumptions, estimates, and judgment calls; it is not an exact science.
\begin{quotation}
  The art of accounting and finance is the art of using limited data to come as close as possible to an accurate description of how well a company is performing.
  \sourceatright{\parencite[4-5]{Berman.Knight2013}}
\end{quotation}

So, the mere existence of gaps, anomalies, and discrepancies is not an indication of fraud. Fraud is deliberate, but gaps, anomalies, and discrepancies can occur because of differing assumptions, simple oversight, recording errors, or (unfortunately)  incompetence.

\subsection{Gaps}\indent%
\label{sec:gaps}

Gaps are where data was expected, but not found.

No significant gaps were found.

\subsection{Anomalies}
\label{sec:anomalies}\indent%

Anomalies are where data was found, but was not what was expected.

\begin{itemize}
  \item It appears that the Rocketship Business Committee only reviews and approves already signed checks in excess of \$100,000. Two things are anomalous here:
  \begin{enumerate}
    \item Rather than reviewing and approving purchase orders (i.e. before signature), the Rocketship Business Committee only retroactively approves checks (after signature). It is not known if those checks have already been sent to their respective payees.
    \item Rather than approving checks for any amount, only those above \$100,000 are reviewed and approved.
  \end{enumerate}
  \item The audited financial statements use a level of materiality (\$300K) that is three times higher than that used by a public school district (LASD) whose budget is half the size of Rocketship's, i.e. Rocketship's level of materiality is 50\% higher than expected..

  \item Administrative expenses, compared to total expenses, seem unusually high. Using functional expenses from the 2021-2022 school year as an example, Rocketship spent \$151,416,849 on educational programs, \$33,683,700 on program support, and \$46,401,574 on management and general expenses.  The management and general expenses are thus approximately 30\% of what was spent on educational programs. In a Business Committee presentation, \textcite[28]{Mukhopadhyay2013}, Rocketship says that the fees they charge individual schools are 35\% of revenue, consisting of a 15\% management fee and a 20\% facility fee, so 30\% is in line with what the Business Committee expects.
  \item The functional expenses that Rocketship has chosen to use in their financial statements differ from the list used in IRS Form 990, Part IX. This makes it nearly impossible to cross check (triangulate) data from Form 900 and the audited annual financial statements.
  \item  For the years ended 2019–2022, accounting expenses were \$166,059 in 2019, more than doubling to \$423,683 in 2020, roughly halving to \$264,784 in 2021, before more than tripling to \$848,221 in 2022. No mention is made of these substantial swings in accounting expenses in the \textit{Notes to Consolidated Financial Statements} for 2022.
  \item In 2022, a total of \$2,635,011c) was spent on travel which is over \$200,000 per month. This represents about 50 cross-country business class flights per month (\$1500 flight with a five day stay at a luxury hotel at \$500/night).  More modest flights and hotel (\$500 + 5×\$300) allow 100 trips per month.  No explanation either for the need for this much travel or nor its cost was provided in the \textit{Notes to Consolidated Financial Statements}, especially in this day and age of Zoom.
  \item The total contributions (i.e. donations, grants) for all Rocketship schools in 2021-2022 is listed as \$7,075,182. The sum total of Object Codes 8980-8999 (Contributions) for all ten Santa Clara County Rocketship Schools is \$3,326,893. These two numbers are clearly not the same, but so are the schools covered. The first consists of all 23 Rocketship schools in the U.S.; the latter consist only of ten schools in Santa Clara County. Considerable work beyond the scope of this investigation would be needed to determine if all the reported contributions agree.

  The SACS Object Codes 8980-8999 are where contributions are recorded in Rocketship's unaudited actuals (and reported to the California Department of Education). The Department of Education makes available a Microsoft Access database with data for specific object codes or groups of object codes for every charter school in California. Summing each school's Object Code 8980-8999 for a test year (YE2020) does not agree with what's is reported on line 11 of Rocketship's Consolidated Statement of Activities for that year, nor does it agree with what was reported on Rocketship's IRS Form 990 that year.

  Several questions remain: Are the differences merely differences in accounting standards of the California Department of Education and the IRS? Or, are the differences choices that Rocketship has made? And if so, why? One final question: the entries for Object Codes 8980-8999 are the only entries which have the form of a positive number under ``restricted'' and an identical number, but negative, under ``unrestricted''. The total is naturally zero.
\end{itemize}

\subsection{Discrepancies}
\label{sec:discrepancies}\indent%

Discrepancies are where two sources of data differ.

The following discrepancies were found:
\begin{itemize}
  \item \textbf{Annual Financial Statements and Form 990s}\\
  The annual audited financial statements have several entries which also appear in the IRS Form 990, the federal tax return for organizations exempt from income tax, i.e. charities, religious organizations, private foundations, some political organizations, and other non-profits. For example, on June 30, 2015, the Consolidated Statement of Financial Position for 2014-2015 shows net assets to be \$10,562,747 (p.3) whereas the Form 990 (2014) show them to be \$13,968,882, a 32\% difference. Analysis of this discrepancy is limited and is insufficient to determine if the difference is the result of differing accounting practices or is a reflection of a more serious underlying problem.

  Similar discrepancies exist for functional expenses, among other categories.
  \item For year 2018–2019, salaries are shown as \$54,294,263 on the audited statement of functional expenses. Yet, adding lines 5 (executive compensation), 7 (other salaries), 8 (pension plan), and 9 (other employee benefits) from Form 990 (2018–2019) yields \$54,516,782 which is close, but not quite the same as the amount shown in the audited statements. Further, it is not even clear that those lines and only those should sum to the same amount as ``Salaries'' in the audited statement of functional expenses.  For example, should pensions be counted as part of salaries? 
\end{itemize}


\section{Issues of Equality and Equity}
\label{sec:issues_equality_equity}\indent%

Ostensibly, issues of equality and equity are at the heart of why Rocketship exists. Their vision is to ``eliminate the achievement gap in our lifetime'' \parencite{RSE2017}.\footnote{Uncharitably, depending on whose lifetime Rocketship is referring to, the elimination of the achievement gap could be 30–60 years out. Included in this span of years is at least one pandemic, one major earthquake, several depressions or recessions, several wars, and numerous government shutdowns.} Their mission is to ``catalyze transformative change in low-income communities through a scalable and sustainable public school model that propels student achievement, develops exceptional educators, and partners with parents who enable high-quality public schools to thrive in their community.''(ibid.) These are laudable goals, but not unique to Rocketship, other charter schools, or even public schools.

Rocketship locates all of its schools in high poverty areas\footnote{areas where 20\% live in poverty or where the median family income less than 80\% of the area median family income\parencite[13-14]{CDFI2020}} where chronically underfunded public schools struggle to provide a quality education to all. Had they not done so, investors would not have been able to take advantage of the NMTC program. Despite being located in high poverty areas, Rocketship claims that its elementary schools are among the best in the nation \parencite{Abousalem2021}. An argument can be made that all Rocketship can actually claim is that their students are among the best standardized test takers in the nation because there is no evidence that Rocketship students who continue their formal education (middle school, high school, college, university) do any better than public schools students. As previously mentioned, \textcite{Lubienski.Lubienski2014} have shown that the NAEP test results of public schools are higher than those of charter schools, all things considered. Of course this does not mean that Rocketship couldn't be an outlier whose students do better in the long run than those of other public or charter schools, but the only evidence that has been presented \parencite{Raymond.etal2023}, like other CREDO publications, has not been well received.\footnote{Stanford University's CREDO (Center for Research on Education Outcomes) makes the case that ``from 2015 to 2019, the typical charter school student in our national sample had reading and math gains that outpaced their peers in the traditional public schools (TPS) they otherwise would have attended''. Reviewing the lastest CREDO report for the NEPC, Joseph Ferrare said, ``[This] CREDO report compares charter school students’ learning in reading and math to students in traditional public schools. The report should be approached with caution by policymakers given the nonexperimental design that renders it unable to fully account for the factors that drive families to choose charter schools. In addition, the report presents its findings using an unconventional metric that makes it difficult to understand the policy implications, potentially misleading policymakers. The magnitude of the main findings fails to meet the minimum threshold experts consider to be a meaningful educational intervention.'' \parencite{Ferrare2023}}

%%% Local Variables:
%%% mode: latex
%%% TeX-master: "Rocketship_Education-An_Exploratory_Public_Policy_Case_Study"
%%% End:
